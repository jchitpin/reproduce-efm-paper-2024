\documentclass[tikz]{standalone}

% Packages
\usepackage{amsmath,amsthm,bm} % math libraries
\usepackage{sankey} % sankey/flow diagram support
\usepackage{graphicx,xcolor} % general image and colour support
\usepackage{caption,subcaption} % caption/subcaptions

% Load custom Tikz code
\usepackage{custom-tikz}

% Hypergraph arrows
% https://tex.stackexchange.com/questions/108064/typesetting-a-directed-hypergraph-in-tikz
\makeatletter
\tikzset{
    @pos/.style={@pos1={#1},@pos2={#1}},
    @ratio/.style={@ratio1={#1},@ratio2={#1}},
    @delta/.style={@delta1={#1},@delta2={#1}},
    @edge/.style={@@edge/.append style={#1}},
    @edge 0/.style={@@edge 0/.append style={#1}},
    @edge 1/.style={@@edge 1/.append style={#1}},
    @edge 2/.style={@@edge 2/.append style={#1}},
    @edge 3/.style={@@edge 3/.append style={#1}},
    @edge 4/.style={@@edge 4/.append style={#1}},
    % and for four:
    @pos1/.store in=\qrr@posA,
    @pos2/.store in=\qrr@posB,
    @ratio1/.store in=\qrr@ratioA,
    @ratio2/.store in=\qrr@ratioB,
    @delta1/.store in=\qrr@deltaA,
    @delta2/.store in=\qrr@deltaB,
    @pos=.5,
    @ratio=.5,
    @delta=.1,
}
\newcommand*{\connectThreeAa}[4][]{
    \begingroup
    \tikzset{#1}
    \coordinate (@aux1) at ($(#2)!\qrr@ratioA!(#3)$);
    \coordinate (@aux2) at ($(#4)!\qrr@posA!(@aux1)$);
    \path (@aux2) edge[@@edge/.try, @@edge 0/.try, @@edge 3/.try] (#4);
    \draw[%
        line width=1.0pt,
        @@edge/.try, @@edge 1/.try,
        -stealth
    ] (@aux2) .. controls ($(#4)!\qrr@posA+\qrr@deltaA!(@aux1)$) .. (#2);
    \draw[%
        line width=1.0pt,
        @@edge/.try, @@edge 2/.try,
        -stealth
    ] (@aux2) .. controls ($(#4)!\qrr@posA+\qrr@deltaA!(@aux1)$) .. (#3);
    \endgroup
}
\newcommand*{\connectThreeBb}[4][]{
    \begingroup
    \tikzset{#1}
    \coordinate (@aux1) at ($(#2)!\qrr@ratioA!(#3)$);
    \coordinate (@aux2) at ($(#4)!\qrr@posA!(@aux1)$);
    \path[line width=1.0pt] (@aux2) edge[@@edge/.try, @@edge 0/.try, @@edge 3/.try,-] (#4);
    \draw[%
        line width=1.0pt,
        @@edge/.try, @@edge 1/.try,
        shorten <=0.035cm
    ] (@aux2) .. controls ($(#4)!\qrr@posA+\qrr@deltaA!(@aux1)$) .. (#2);
    \draw[%
        line width=1.0pt,
        @@edge/.try, @@edge 2/.try, stealth-
    ] (@aux2) .. controls ($(#4)!\qrr@posA+\qrr@deltaA!(@aux1)$) .. (#3);
    \endgroup
}
% \renewcommand*{\connectThree}[4][]{\connectFour[#1, @@edge 4/.style={draw=none}, @ratio2=0]{#2}{#3}{#4}{0,0}}

\newcommand*{\connectFour}[5][]{
    \begingroup
    \tikzset{#1}
    \coordinate (@aux1a) at ($(#2)!\qrr@ratioA!(#3)$);
    \coordinate (@aux1b) at ($(#4)!\qrr@ratioB!(#5)$);
    \coordinate (@aux2a) at ($(@aux1b)!\qrr@posA!(@aux1a)$);
    \coordinate (@aux2b) at ($(@aux1a)!\qrr@posB!(@aux1b)$);
    \path (@aux2a) edge[@@edge/.try,@@edge 0/.try] (@aux2b);
    \draw[@@edge/.try,@@edge 1/.try] (@aux2a) .. controls ($(@aux1b)!\qrr@posA+\qrr@deltaA!(@aux1a)$) .. (#2);
    \draw[@@edge/.try,@@edge 2/.try] (@aux2a) .. controls ($(@aux1b)!\qrr@posA+\qrr@deltaA!(@aux1a)$) .. (#3);
    \draw[@@edge/.try,@@edge 3/.try] (@aux2b) .. controls ($(@aux1a)!\qrr@posB+\qrr@deltaB!(@aux1b)$) .. (#4);
    \draw[@@edge/.try,@@edge 4/.try] (@aux2b) .. controls ($(@aux1a)!\qrr@posB+\qrr@deltaB!(@aux1b)$) .. (#5);
    \draw[help lines] (@aux1a) -- (@aux1b) node[midway,above,sloped,font=\tiny,shape=rectangle,inner xsep=+0pt,draw=none,align=center,fill=white,fill opacity=.75,outer ysep=\pgflinewidth,text opacity=1] {ratio: \qrr@ratioA/\qrr@ratioB\\pos: \qrr@posA/\qrr@posB\\delta: \qrr@deltaA/\qrr@deltaB};
    \endgroup
}

% Updated with colour!
\makeatletter
\tikzset{
    @pos/.style={@pos1={#1},@pos2={#1}},
    @ratio/.style={@ratio1={#1},@ratio2={#1}},
    @delta/.style={@delta1={#1},@delta2={#1}},
    @edge/.style={@@edge/.append style={#1}},
    @edge 0/.style={@@edge 0/.append style={#1}},
    @edge 1/.style={@@edge 1/.append style={#1}},
    @edge 2/.style={@@edge 2/.append style={#1}},
    @edge 3/.style={@@edge 3/.append style={#1}},
    @edge 4/.style={@@edge 4/.append style={#1}},
    % and for four:
    @pos1/.store in=\qrr@posA,
    @pos2/.store in=\qrr@posB,
    @ratio1/.store in=\qrr@ratioA,
    @ratio2/.store in=\qrr@ratioB,
    @delta1/.store in=\qrr@deltaA,
    @delta2/.store in=\qrr@deltaB,
    @pos=.5,
    @ratio=.5,
    @delta=.1,
}
\newcommand*{\connectThreeA}[4][]{
    \begingroup
    \tikzset{#1}
    \coordinate (@aux1) at ($(#2)!\qrr@ratioA!(#3)$);
    \coordinate (@aux2) at ($(#4)!\qrr@posA!(@aux1)$);
    \path (@aux2) edge[@@edge/.try, @@edge 0/.try, @@edge 3/.try] (#4);
    \draw[%
        corange,
        line width=1.0pt,
        @@edge/.try, @@edge 1/.try,
        -stealth
    ] (@aux2) .. controls ($(#4)!\qrr@posA+\qrr@deltaA!(@aux1)$) .. (#2);
    \draw[%
        corange,
        line width=1.0pt,
        @@edge/.try, @@edge 2/.try,
        -stealth
    ] (@aux2) .. controls ($(#4)!\qrr@posA+\qrr@deltaA!(@aux1)$) .. (#3);
    \endgroup
}
\newcommand*{\connectThreeB}[4][]{
    \begingroup
    \tikzset{#1}
    \coordinate (@aux1) at ($(#2)!\qrr@ratioA!(#3)$);
    \coordinate (@aux2) at ($(#4)!\qrr@posA!(@aux1)$);
    \path[line width=1.0pt] (@aux2) edge[@@edge/.try, @@edge 0/.try, @@edge 3/.try,-] (#4);
    \draw[%
        corange,
        line width=1.0pt,
        @@edge/.try, @@edge 1/.try,
        shorten <=0.035cm
    ] (@aux2) .. controls ($(#4)!\qrr@posA+\qrr@deltaA!(@aux1)$) .. (#2);
    \draw[%
        corange,
        line width=1.0pt,
        @@edge/.try, @@edge 2/.try, stealth-
    ] (@aux2) .. controls ($(#4)!\qrr@posA+\qrr@deltaA!(@aux1)$) .. (#3);
    \endgroup
}
% \renewcommand*{\connectThree}[4][]{\connectFour[#1, @@edge 4/.style={draw=none}, @ratio2=0]{#2}{#3}{#4}{0,0}}

\newcommand*{\connectFour}[5][]{
    \begingroup
    \tikzset{#1}
    \coordinate (@aux1a) at ($(#2)!\qrr@ratioA!(#3)$);
    \coordinate (@aux1b) at ($(#4)!\qrr@ratioB!(#5)$);
    \coordinate (@aux2a) at ($(@aux1b)!\qrr@posA!(@aux1a)$);
    \coordinate (@aux2b) at ($(@aux1a)!\qrr@posB!(@aux1b)$);
    \path (@aux2a) edge[@@edge/.try,@@edge 0/.try] (@aux2b);
    \draw[@@edge/.try,@@edge 1/.try] (@aux2a) .. controls ($(@aux1b)!\qrr@posA+\qrr@deltaA!(@aux1a)$) .. (#2);
    \draw[@@edge/.try,@@edge 2/.try] (@aux2a) .. controls ($(@aux1b)!\qrr@posA+\qrr@deltaA!(@aux1a)$) .. (#3);
    \draw[@@edge/.try,@@edge 3/.try] (@aux2b) .. controls ($(@aux1a)!\qrr@posB+\qrr@deltaB!(@aux1b)$) .. (#4);
    \draw[@@edge/.try,@@edge 4/.try] (@aux2b) .. controls ($(@aux1a)!\qrr@posB+\qrr@deltaB!(@aux1b)$) .. (#5);
    \draw[help lines] (@aux1a) -- (@aux1b) node[midway,above,sloped,font=\tiny,shape=rectangle,inner xsep=+0pt,draw=none,align=center,fill=white,fill opacity=.75,outer ysep=\pgflinewidth,text opacity=1] {ratio: \qrr@ratioA/\qrr@ratioB\\pos: \qrr@posA/\qrr@posB\\delta: \qrr@deltaA/\qrr@deltaB};
    \endgroup
}

\makeatother

% Colour definitions
\definecolor{cred}{HTML}{ED1C24}
\definecolor{cgrey}{HTML}{7F7F7F}
\definecolor{cblue}{HTML}{00A2E8}
\definecolor{cgreen}{HTML}{22B14C}
\definecolor{cyellow}{HTML}{FFF200}
\definecolor{corange}{HTML}{EA7904}
\definecolor{cpurple}{HTML}{9100FC}
\definecolor{efm1}{HTML}{017EC3}
\definecolor{efm2}{HTML}{F31E26}
\definecolor{efm3}{HTML}{019E5E}
\definecolor{efm4}{HTML}{FBA61D}
\definecolor{efm5}{HTML}{916237}

% Font style
\renewcommand{\familydefault}{\sfdefault}

\begin{document}

    % Figure
    \begin{tikzpicture}[]
        % (a) metabolic network
        \node[labelFont] (Aa) {A};
\node[labelFont,right=1.5cm of Aa] (Ba) {B};
\node[labelFont,right=1.5cm of Ba] (Ca) {C};
\node[labelFont,right=1.5cm of Ca] (Da) {D};
\node[labelFont,yshift=-1.0cm] (G1a) at ($(Ba)!0.5!(Ca)$) {G\textsubscript{1}};
\node[labelFont,yshift=-1.75cm] (G2a) at ($(Ba)!0.5!(Ca)$) {G\textsubscript{2}};
\node[labelFont,right=1.0cm of G1a,yshift=0.25cm] (Ea) {E};
\node[labelFont,right=1.0cm of G2a,yshift=0.25cm] (Fa) {F};
\node[font=\normalsize,right=0.1cm of G1a,yshift=0.2cm,xshift=-0.15cm] () {$v_8$};
\node[font=\normalsize, left=0.1cm of G1a,yshift=0.2cm,xshift=+0.15cm] () {$v_9$};
\node[left=0.2cm of Aa] (AEa) {};
\node[right=0.2cm of Da] (DEa) {};
\node[right=0.2cm of Ea] (EEa) {};
\node[right=0.2cm of Fa] (FEa) {};
\node[left=0.5cm of Aa] (left) {};
\node[right=0.5cm of Da] (right) {};

\draw[networkArrow] ([yshift=+0.00cm]AEa.east) to node[midway,above,font=\normalsize,yshift=+0.0cm] {$v_1$} ([yshift=+0.00cm]Aa.west);
\draw[networkArrow] ([yshift=+0.00cm]Aa.east) to node[midway,above,font=\normalsize,yshift=+0.0cm] {$v_2$} ([yshift=+0.00cm]Ba.west);
\draw[networkArrow] ([yshift=+0.00cm]Ba.east) to node[midway,above,font=\normalsize,yshift=+0.0cm] {$v_3$} ([yshift=+0.00cm]Ca.west);
\draw[networkArrow] ([yshift=+0.00cm]Ca.east) to node[midway,above,font=\normalsize,yshift=+0.0cm] {$v_4$} ([yshift=+0.00cm]Da.west);
\draw[networkArrow] ([yshift=+0.00cm]Da.east) to node[midway,above,font=\normalsize,yshift=+0.0cm] {$v_5$} ([yshift=+0.00cm]DEa.west);
\draw[networkArrow] ([yshift=+0.00cm]EEa.west) to node[midway,above,font=\normalsize,yshift=+0.0cm] {$v_6$} ([yshift=+0.00cm]Ea.east);
\draw[networkArrow] ([yshift=+0.00cm]Fa.east) to node[midway,above,font=\normalsize,yshift=+0.0cm] {$v_7$} ([yshift=+0.00cm]FEa.west);
\connectThreeAa[%
    @ratio=3.5,
    @pos=0.0,
    @delta=0.3,
    @edge 1=->,
    @edge 2=->
]{G1a.east}{G2a.east}{Ca.south}
\connectThreeBb[%
    @ratio=3.5,
    @pos=0.0,
    @delta=0.3,
    @edge 3=->
]{G1a.west}{G2a.west}{Ba.south}
\draw[-stealth, line width=1.0pt] (Ea.west) to[out=210, in=165, looseness=2.75] node[] {} (Fa.west);

\node[fit=(left) (right) (AEa) (Aa) (Ba) (Ca) (Da) (DEa) (Ea) (EEa) (Fa) (FEa) (G1a) (G2a), draw=black,inner sep=5pt, line width=1.0pt] (metabolic_flux_network) {};
\node[above=0.1cm of metabolic_flux_network, labelFont, yshift=-0.1cm] () {Metabolic flux network};

% Subcaption label
\node[labelFont,left=4.65cm of metabolic_flux_network.north west,yshift=1.0cm,anchor=south,font=\huge] () {(a)};



        % (b) Metabolic graph
        \node[line width=0.75pt, labelFont,below=5.5cm of Aa,xshift=-5.0cm,draw] (A) {\phantom{A}};
\node[carbon2,above=0.1cm of A,yshift=-0.725cm,draw=none] (Ac1) {};
\node[carbon2,below=0.1cm of A,yshift=+0.725cm,draw=none] (Ac2) {};
\node[labelFont,above=0.1cm of A,yshift=-0.1cm] () {A};

\node[line width=0.75pt,labelFont,right=1.5cm of A,draw] (B) {\phantom{B}};
\node[carbon2,above=0.1cm of B,yshift=-0.725cm,draw=none] (Bc1) {};
\node[carbon2,below=0.1cm of B,yshift=+0.725cm,draw=none] (Bc2) {};
\node[labelFont,above=0.1cm of B,yshift=-0.1cm] () {B};

\node[line width=0.75pt,labelFont,right=1.5cm of B,draw] (C) {\phantom{C}};
\node[carbon2,above=0.1cm of C,yshift=-0.725cm,draw=none] (Cc1) {};
\node[carbon2,below=0.1cm of C,yshift=+0.725cm,draw=none] (Cc2) {};
\node[labelFont,above=0.1cm of C,yshift=-0.1cm] () {C};

\node[line width=0.75pt,labelFont,right=1.5cm of C,draw] (D) {\phantom{D}};
\node[carbon2,above=0.1cm of D,yshift=-0.725cm,draw=none] (Dc1) {};
\node[carbon2,below=0.1cm of D,yshift=+0.725cm,draw=none] (Dc2) {};
\node[labelFont,above=0.1cm of D,yshift=-0.1cm] () {D};

\node[line width=0.75pt,labelFont,yshift=-1.0cm,draw=none] (G1) at ($(B)!0.5!(C)$) {\phantom{G}};
\node[carbon2,draw=none] (G1c) at (G1.center) {};

\node[line width=0.75pt,labelFont,yshift=-1.75cm,draw] (G2) at ($(B)!0.5!(C)$) {\phantom{G}};
\node[carbon2,draw=none] (G2c) at (G2.center) {};
\node[labelFont,below=0.1cm of G2,yshift=+0.1cm] () {G};

\node[line width=0.75pt,labelFont,right=1.5cm of G1,yshift=-0.25cm,draw] (E) {\phantom{E}};
\node[neutral2,draw=none] (Ed) at (E.center) {};
\node[labelFont,above=0.1cm of E,yshift=-0.1cm] () {E};

\node[line width=0.75pt,labelFont,right=1.5cm of G2,yshift=-0.5cm,draw] (F) {\phantom{F}};
\node[neutral2,draw=none] (Fd) at (F.center) {};
\node[labelFont,below=0.1cm of F,yshift=+0.1cm] () {F};

\node[labelFont,left=0.2cm of A] (AE) {\phantom{A}};
\node[carbon2,above=0.1cm of AE,yshift=-0.725cm,draw=none] (AEc1) {};
\node[carbon2,below=0.1cm of AE,yshift=+0.725cm,draw=none] (AEc2) {};

\node[labelFont,right=0.2cm of D] (DE) {\phantom{D}};
\node[carbon2,above=0.1cm of DE,yshift=-0.725cm,draw=none] (DEc1) {};
\node[carbon2,below=0.1cm of DE,yshift=+0.725cm,draw=none] (DEc2) {};

\node[right=0.2cm of E] (EE) {};
\node[right=0.2cm of F] (FE) {};

\node[left=0.5cm of A] (left) {};
\node[right=0.5cm of D] (right) {};

\node[above=0.5cm of A] (pad_up) {};
\node[below=0.5cm of F] (pad_down) {};

\draw[networkArrow] ([xshift=+0.0cm]EE.west) to node[midway,above,font=\normalsize,yshift=+0.0cm] {} ([xshift=+0.13cm]Ed.east);
\draw[networkArrow] ([xshift=+0.13cm]Fd.east) to node[midway,above,font=\normalsize,yshift=+0.0cm] {} ([xshift=-0.0cm]FE.west);
\draw[networkArrow] ([xshift=-0.01cm]AE.east) to node[midway,above,font=\normalsize,yshift=+0.0cm] {} ([xshift=-0.0cm]A.west);
\draw[networkArrow] ([xshift=+0.0cm]A.east) to node[midway,above,font=\normalsize,yshift=+0.0cm] {} ([xshift=-0.0cm]B.west);
\draw[networkArrow] ([xshift=+0.0cm]B.east) to node[midway,above,font=\normalsize,yshift=+0.0cm] {} ([xshift=-0.0cm]C.west);
\draw[networkArrow] ([xshift=+0.0cm]C.east) to node[midway,above,font=\normalsize,yshift=+0.0cm] {} ([xshift=-0.0cm]D.west);
\draw[networkArrow] ([xshift=+0.0cm]D.east) to node[midway,above,font=\normalsize,yshift=+0.0cm] {} ([xshift=+0.01cm]DE.west);
\draw[networkArrow] ([xshift=+0.0cm]E.south) to node[midway,above,font=\normalsize,yshift=+0.0cm] {} ([xshift=+0.00cm]F.north);

\draw[networkArrow] ([xshift=+0.0cm]C.south) to node[midway,above,font=\normalsize,yshift=+0.0cm] {} ([xshift=+0.00cm]G2.north east);
\draw[networkArrow] ([xshift=+0.0cm]G2.north west) to node[midway,above,font=\normalsize,yshift=+0.0cm] {} ([xshift=+0.00cm]B.south);
\draw[networkArrow] ([xshift=+0.0cm]C.south) to node[midway,above,font=\normalsize,yshift=+0.0cm] {} ([xshift=+0.00cm]E.north west);
\draw[networkArrow] ([xshift=+0.0cm]C.south) to node[midway,above,font=\normalsize,yshift=+0.0cm] {} ([xshift=+0.00cm]F.north west);


\node[fit=(left) (right) (A) (B) (C) (D) (E) (F) (G1) (G2) (pad_up) (pad_down), draw=black,inner sep=5pt, line width=1.0pt] (metabolic_graph) {};
\node[above=0.1cm of metabolic_graph, labelFont,yshift=-0.1cm] () {Metabolic graph};

% Subcaption label
\node[labelFont,left=0.2cm of metabolic_graph.north west,yshift=1.0cm,anchor=south west,font=\huge] () {(b)};



        % (c) Atomic transition/flux graph
        \node[line width=0.75pt, labelFont,below=5.5cm of Aa,xshift=5.0cm,draw] (A) {\phantom{A}};
\node[carbon2,above=0.1cm of A,yshift=-0.725cm] (Ac1) {};
\node[carbon2,below=0.1cm of A,yshift=+0.725cm] (Ac2) {};
\node[labelFont,above=0.1cm of A,yshift=-0.1cm] () {A};

\node[line width=0.75pt,labelFont,right=1.5cm of A,draw] (B) {\phantom{B}};
\node[carbon2,above=0.1cm of B,yshift=-0.725cm] (Bc1) {};
\node[carbon2,below=0.1cm of B,yshift=+0.725cm] (Bc2) {};
\node[labelFont,above=0.1cm of B,yshift=-0.1cm] () {B};

\node[line width=0.75pt,labelFont,right=1.5cm of B,draw] (C) {\phantom{C}};
\node[carbon2,above=0.1cm of C,yshift=-0.725cm] (Cc1) {};
\node[carbon2,below=0.1cm of C,yshift=+0.725cm] (Cc2) {};
\node[labelFont,above=0.1cm of C,yshift=-0.1cm] () {C};

\node[line width=0.75pt,labelFont,right=1.5cm of C,draw] (D) {\phantom{D}};
\node[carbon2,above=0.1cm of D,yshift=-0.725cm] (Dc1) {};
\node[carbon2,below=0.1cm of D,yshift=+0.725cm] (Dc2) {};
\node[labelFont,above=0.1cm of D,yshift=-0.1cm] () {D};

\node[line width=0.75pt,labelFont,yshift=-1.0cm,draw] (G1) at ($(B)!0.5!(C)$) {\phantom{G}};
\node[carbon2] (G1c) at (G1.center) {};

\node[line width=0.75pt,labelFont,yshift=-1.75cm,draw] (G2) at ($(B)!0.5!(C)$) {\phantom{G}};
\node[carbon2] (G2c) at (G2.center) {};
\node[labelFont,below=0.1cm of G2,yshift=+0.1cm] () {G\textsubscript{1,2}};

\node[line width=0.75pt,labelFont,right=1.5cm of G1,yshift=-0.25cm,draw] (E) {\phantom{E}};
\node[carbon2] (Ed) at (E.center) {};
\node[labelFont,above=0.1cm of E,yshift=-0.1cm] () {E};

\node[line width=0.75pt,labelFont,right=1.5cm of G2,yshift=-0.5cm,draw] (F) {\phantom{F}};
\node[carbon2] (Fd) at (F.center) {};
\node[labelFont,below=0.1cm of F,yshift=+0.1cm] () {F};

\node[labelFont,left=0.2cm of A] (AE) {\phantom{A}};
\node[carbon2,above=0.1cm of AE,yshift=-0.725cm,draw=none] (AEc1) {};
\node[carbon2,below=0.1cm of AE,yshift=+0.725cm,draw=none] (AEc2) {};

\node[labelFont,right=0.2cm of D] (DE) {\phantom{D}};
\node[carbon2,above=0.1cm of DE,yshift=-0.725cm,draw=none] (DEc1) {};
\node[carbon2,below=0.1cm of DE,yshift=+0.725cm,draw=none] (DEc2) {};

\node[right=0.2cm of E] (EE) {};
\node[right=0.2cm of F] (FE) {};

\node[left=0.5cm of A] (left) {};
\node[right=0.5cm of D] (right) {};

\node[above=0.5cm of A] (pad_up) {};
\node[below=0.5cm of F] (pad_down) {};

\draw[networkArrow] ([xshift=+0.17cm]AEc1.east) to node[above,font=\normalsize,xshift=+0.0cm] {} ([xshift=-0.2cm]Ac1.west);
\draw[networkArrow] ([xshift=+0.17cm]AEc2.east) to node[midway,above,font=\normalsize,xshift=-0.1cm] {$v_1$} ([xshift=-0.2cm]Ac2.west);
\draw[networkArrow] ([xshift=+0.2cm]Ac1.east) to node[midway,above,font=\normalsize,yshift=+0.0cm] {} ([xshift=-0.55cm]Bc1.west);
\draw[networkArrow] ([xshift=+0.2cm]Ac2.east) to node[midway,above,font=\normalsize,yshift=+0.0cm] {$v_2$} ([xshift=-0.55cm]Bc2.west);
\draw[networkArrow] ([xshift=+0.2cm]Bc1.east) to node[midway,above,font=\normalsize,yshift=+0.0cm] {} ([xshift=-0.2cm]Cc1.west);
\draw[networkArrow] ([xshift=+0.2cm]Bc2.east) to node[midway,above,font=\normalsize,yshift=+0.0cm] {$v_3$} ([xshift=-0.2cm]Cc2.west);
\draw[networkArrow] ([xshift=+0.55cm]Cc1.east) to node[midway,above,font=\normalsize,yshift=+0.0cm] {} ([xshift=-0.2cm]Dc1.west);
\draw[networkArrow] ([xshift=+0.55cm]Cc2.east) to node[midway,above,font=\normalsize,yshift=+0.0cm] {$v_4$} ([xshift=-0.2cm]Dc2.west);
\draw[networkArrow] ([xshift=+0.2cm]Dc1.east) to node[midway,above,font=\normalsize,yshift=+0.0cm] {} ([xshift=-0.2cm]DEc1.west);
\draw[networkArrow] ([xshift=+0.2cm]Dc2.east) to node[midway,above,font=\normalsize,xshift=+0.1cm] {$v_5$} ([xshift=-0.2cm]DEc2.west);

\draw[-stealth, line width=1.0pt] ([xshift=+0.20cm]Cc2.east) to[out=-15, in=0, looseness=1.0] node[] {} (G2.east);
\draw[-stealth, line width=1.0pt] ([xshift=+0.20cm]Cc1.east) to[out=-15, in=0, looseness=1.0] node[midway,below] {$v_8$} (G1.east);
\draw[-stealth, line width=1.0pt] (G2.west) to[out=180, in=180, looseness=1.0] node[] {} ([xshift=-0.20cm]Bc2.west);
\draw[-stealth, line width=1.0pt] (G1.west) to[out=180, in=180, looseness=1.0] node[midway,below] {$v_9$} ([xshift=-0.20cm]Bc1.west);

%\draw[-stealth, line width=1.0pt] ([xshift=-0.15cm]Ed.west) to[out=210, in=165, looseness=2.75] node[] {} ([xshift=-0.15cm]Fd.west);
\draw[-stealth, line width=1.0pt] (E.south) to node[midway,right] {$v_8$} (F.north);

\draw[networkArrow] ([xshift=+0.0cm]EE.west) to node[midway,font=\normalsize,xshift=+0.3cm] {$v_6$} ([xshift=+0.13cm]Ed.east);
\draw[networkArrow] ([xshift=+0.13cm]Fd.east) to node[midway,font=\normalsize,xshift=+0.3cm] {$v_7$} ([xshift=-0.0cm]FE.west);


\node[fit=(left) (right) (A) (B) (C) (D) (E) (F) (G1) (G2) (pad_up) (pad_down), draw=black,inner sep=5pt, line width=1.0pt] (atom_transition_graph) {};
%\node[above=0.1cm of atom_transition_graph, labelFont,yshift=-0.1cm] () {Atom transition graph};
\node[above=0.1cm of atom_transition_graph, labelFont,yshift=-0.1cm,align=center] () {Atomic transition/flux graph};

% Subcaption label
\node[labelFont,left=0.2cm of atom_transition_graph.north west,yshift=1.0cm,anchor=south west,font=\huge] () {(c)};



        % (d) EFMs
        \node[labelFont,below=12.0cm of Aa,xshift=-5.0cm,draw] (A) {A};
\node[labelFont,right=1.5cm of A] (Ba) {B};
\node[labelFont,right=1.5cm of Ba] (Ca) {C};
\node[labelFont,right=1.5cm of Ca] (Da) {D};
\node[labelFont,yshift=-1.0cm] (G1a) at ($(Ba)!0.5!(Ca)$) {G\textsubscript{1}};
\node[labelFont,yshift=-1.75cm] (G2a) at ($(Ba)!0.5!(Ca)$) {G\textsubscript{2}};
\node[labelFont,right=1.0cm of G1a,yshift=0.25cm] (Ea) {E};
\node[labelFont,right=1.0cm of G2a,yshift=0.25cm] (Fa) {F};
\node[font=\normalsize,right=0.1cm of G1a,yshift=0.2cm,xshift=-0.15cm] () {};
\node[font=\normalsize, left=0.1cm of G1a,yshift=0.2cm,xshift=+0.15cm] () {};
\node[left=0.2cm of A] (AEa) {};
\node[right=0.2cm of Da] (DEa) {};
\node[right=0.2cm of Ea] (EEa) {};
\node[right=0.2cm of Fa] (FEa) {};
\node[left=0.5cm of A] (left) {};
\node[right=0.5cm of Da] (right) {};

\draw[networkArrow,cpurple] ([yshift=+0.10cm]AEa.east) to node[midway,font=\normalsize,xshift=-0.45cm] {$w_1$} ([yshift=+0.10cm]A.west);
\draw[networkArrow,cpurple] ([yshift=+0.10cm]A.east) to node[midway,above,font=\normalsize,yshift=+0.0cm] {} ([yshift=+0.10cm]Ba.west);
\draw[networkArrow,cpurple] ([yshift=+0.10cm]Ba.east) to node[midway,above,font=\normalsize,yshift=+0.0cm] {} ([yshift=+0.10cm]Ca.west);
\draw[networkArrow,cpurple] ([yshift=+0.10cm]Ca.east) to node[midway,above,font=\normalsize,yshift=+0.0cm] {} ([yshift=+0.10cm]Da.west);
\draw[networkArrow,cpurple] ([yshift=+0.10cm]Da.east) to node[midway,above,font=\normalsize,yshift=+0.0cm] {} ([yshift=+0.10cm]DEa.west);

\draw[networkArrow,corange] ([yshift=-0.10cm]Ba.east) to node[midway,above,font=\normalsize,yshift=+0.0cm] {} ([yshift=-0.10cm]Ca.west);

\draw[networkArrow,corange] ([yshift=+0.00cm]EEa.west) to node[midway,font=\normalsize,xshift=+0.45cm] {$w_2$} ([yshift=+0.00cm]Ea.east);
\draw[networkArrow,corange] ([yshift=+0.00cm]Fa.east) to node[midway,above,font=\normalsize,yshift=+0.0cm] {} ([yshift=+0.00cm]FEa.west);
\connectThreeA[%
    @ratio=3.5,
    @pos=0.0,
    @delta=0.3,
    @edge 1=->,
    @edge 2=->
]{G1a.east}{G2a.east}{Ca.south}
\connectThreeB[%
    @ratio=3.5,
    @pos=0.0,
    @delta=0.3,
    @edge 3=->
]{G1a.west}{G2a.west}{Ba.south}
\draw[-stealth, line width=1.0pt,corange] (Ea.west) to[out=210, in=165, looseness=2.75] node[] {} (Fa.west);

\node[fit=(left) (right) (AEa) (A) (Ba) (Ca) (Da) (DEa) (Ea) (EEa) (Fa) (FEa) (G1a) (G2a), draw=black,inner sep=5pt, line width=1.0pt] (metabolic_flux_network) {};
\node[above=0.1cm of metabolic_flux_network, labelFont, yshift=-0.1cm,align=center] () {Molecular elementary flux modes};

% Subcaption label
\node[labelFont,left=0.2cm of metabolic_flux_network.north west,yshift=1.0cm,anchor=south west,font=\huge] () {(d)};



        % (e) AEFMs
        %\node[line width=0.75pt, labelFont,below=5.5cm of Aa,xshift=-5.0cm,draw] (A) {\phantom{A}};
\node[line width=0.75pt,labelFont,right=9.0cm of Aa,draw,below=12.0cm of Aa,xshift=+5.0cm] (Aa) {\phantom{A}};
\node[carbon2,above=0.1cm of Aa,yshift=-0.725cm] (Aac1) {};
\node[carbon2,below=0.1cm of Aa,yshift=+0.725cm] (Aac2) {};

\node[line width=0.75pt,labelFont,right=1.5cm of Aa,draw] (Ba) {\phantom{B}};
\node[carbon2,above=0.1cm of Ba,yshift=-0.725cm] (Bac1) {};
\node[carbon2,below=0.1cm of Ba,yshift=+0.725cm] (Bac2) {};

\node[line width=0.75pt,labelFont,right=1.5cm of Ba,draw] (Ca) {\phantom{C}};
\node[carbon2,above=0.1cm of Ca,yshift=-0.725cm] (Cac1) {};
\node[carbon2,below=0.1cm of Ca,yshift=+0.725cm] (Cac2) {};

\node[line width=0.75pt,labelFont,right=1.5cm of Ca,draw] (Da) {\phantom{D}};
\node[carbon2,above=0.1cm of Da,yshift=-0.725cm] (Dac1) {};
\node[carbon2,below=0.1cm of Da,yshift=+0.725cm] (Dac2) {};

\node[line width=0.75pt,labelFont,yshift=-1.0cm,draw] (G1a) at ($(Ba)!0.5!(Ca)$) {\phantom{G}};
\node[carbon2] at (G1a.center) {};

\node[line width=0.75pt,labelFont,yshift=-1.79cm,draw] (G2a) at ($(Ba)!0.5!(Ca)$) {\phantom{G}};
\node[carbon2] at (G2a.center) {};

\node[line width=0.75pt,labelFont,right=1.75cm of G1a,yshift=0.25cm,draw] (Ea) {\phantom{E}};
\node[carbon2] at (Ea) {};

\node[line width=0.75pt,labelFont,right=1.75cm of G2a,yshift=0.0cm,draw] (Fa) {\phantom{F}};
\node[carbon2] at (Fa) {};

%\node[font=\normalsize,right=0.1cm of G1a,yshift=0.2cm,xshift=-0.15cm] () {$v_8$};
%\node[font=\normalsize, left=0.1cm of G1a,yshift=0.2cm,xshift=+0.15cm] () {$v_9$};
\node[left=0.2cm of Aa] (AEa) {};
\node[right=0.2cm of Da] (DEa) {};
\node[right=0.2cm of Ea] (EEa) {};
\node[right=0.2cm of Fa] (FEa) {};
\node[left=0.5cm of Aa] (left) {};
\node[right=0.5cm of Da] (right) {};

\node[below=1.0cm of Ba,efm3] () {$x_4$};
\node[below=0.2cm of Ba,efm4] () {$x_3$};

\draw[networkArrow,efm1] ([yshift=+0.25cm]AEa.east) to node[midway,font=\normalsize,xshift=-0.45cm,yshift=+0.0cm] {$x_{1}$} ([yshift=+0.25cm]Aa.west);
\draw[networkArrow,efm1] ([yshift=+0.25cm]Aa.east) to node[midway,above,font=\normalsize,yshift=+0.0cm] {} ([yshift=+0.25cm]Ba.west);
\draw[networkArrow,efm1] ([yshift=+0.25cm]Ba.east) to node[midway,above,font=\normalsize,yshift=+0.0cm] {} ([yshift=+0.25cm]Ca.west);
\draw[networkArrow,efm1] ([yshift=+0.25cm]Ca.east) to node[midway,above,font=\normalsize,yshift=+0.0cm] {} ([yshift=+0.25cm]Da.west);
\draw[networkArrow,efm1] ([yshift=+0.25cm]Da.east) to node[midway,above,font=\normalsize,yshift=+0.0cm] {} ([yshift=+0.25cm]DEa.west);

\draw[networkArrow,efm2] ([yshift=-0.05cm]AEa.east) to node[midway,font=\normalsize,xshift=-0.45cm,yshift=-0.0cm] {$x_{2}$} ([yshift=-0.05cm]Aa.west);
\draw[networkArrow,efm2] ([yshift=-0.05cm]Aa.east) to node[midway,above,font=\normalsize,yshift=+0.0cm] {} ([yshift=-0.05cm]Ba.west);
\draw[networkArrow,efm3] ([yshift=+0.10cm]Ba.east) to node[midway,above,font=\normalsize,yshift=+0.0cm] {} ([yshift=+0.10cm]Ca.west);
\draw[networkArrow,efm2] ([yshift=-0.05cm]Ba.east) to node[midway,above,font=\normalsize,yshift=+0.0cm] {} ([yshift=-0.05cm]Ca.west);
\draw[networkArrow,efm4] ([yshift=-0.20cm]Ba.east) to node[midway,above,font=\normalsize,yshift=+0.0cm] {} ([yshift=-0.20cm]Ca.west);
\draw[networkArrow,efm2] ([yshift=-0.05cm]Ca.east) to node[midway,above,font=\normalsize,yshift=+0.0cm] {} ([yshift=-0.05cm]Da.west);
\draw[networkArrow,efm2] ([yshift=-0.05cm]Da.east) to node[midway,above,font=\normalsize,yshift=+0.0cm] {} ([yshift=-0.05cm]DEa.west);

\node[right=0.5cm of Cac1,yshift=-0.5cm] (ph1) {};
\node[right=0.3cm of Cac1,yshift=-0.5cm] (ph2) {};
\node[left=0.5cm of Bac1,yshift=-0.5cm] (ph3) {};
\node[left=0.3cm of Bac1,yshift=-0.5cm] (ph4) {};

\draw[networkArrow,-,rounded corners=5pt,efm3] ([yshift=+0.10cm]Ca.east) -| (ph1.center) {};
\draw[networkArrow,  rounded corners=5pt,efm3] (ph1.center) |- (G2a) {};
\draw[networkArrow,-,rounded corners=3pt,efm4] ([yshift=+0.00cm]G1a.west) -| (ph4.center) {};
\draw[networkArrow,  rounded corners=3pt,efm4] (ph4.center) |- ([yshift=-0.20cm]Ba.west) {};

\draw[networkArrow,-,rounded corners=3pt,efm4] ([yshift=-0.20cm]Ca.east) -| (ph2.center) {};
\draw[networkArrow,  rounded corners=3pt,efm4] (ph2.center) |- (G1a) {};
\draw[networkArrow,-,rounded corners=5pt,efm3] ([yshift=+0.00cm]G2a.west) -| (ph3.center) {};
\draw[networkArrow,  rounded corners=5pt,efm3] (ph3.center) |- ([yshift=+0.10cm]Ba.west) {};

\draw[networkArrow,efm5] ([yshift=+0.00cm]EEa.west) to node[midway,font=\normalsize,xshift=+0.45cm] {$y_1$} ([yshift=+0.00cm]Ea.east);
\draw[networkArrow,efm5] ([yshift=+0.00cm]Ea.south) to node[midway,above,font=\normalsize,yshift=+0.0cm] {} ([yshift=+0.00cm]Fa.north);
\draw[networkArrow,efm5] ([yshift=+0.00cm]Fa.east) to node[midway,above,font=\normalsize,yshift=+0.0cm] {} ([yshift=+0.00cm]FEa.west);

\node[fit=(left) (right) (AEa) (Aa) (Ba) (Ca) (Da) (DEa) (Ea) (EEa) (Fa) (FEa) (G1a) (G2a), draw=black,inner sep=5pt, line width=1.0pt] (atom_transition_graph) {};
\node[above=0.1cm of atom_transition_graph, labelFont, yshift=-0.1cm,align=center] () {Atomic elementary flux modes};

% Subcaption label
\node[labelFont,left=0.2cm of atom_transition_graph.north west,yshift=1.0cm,anchor=south west,font=\huge] () {(e)};



    \end{tikzpicture}
\end{document}
