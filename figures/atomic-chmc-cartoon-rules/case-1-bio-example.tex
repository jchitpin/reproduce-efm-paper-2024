\documentclass{standalone}
\usepackage[margin=1.00in]{geometry}
\usepackage{xcolor}
\usepackage{caption}
\usepackage{subcaption}
\usepackage{chemfig}
\usepackage{tikz}
\usetikzlibrary{positioning,calc}

% Define colours
\definecolor{cred}{HTML}{ED1C24}
\definecolor{cgrey}{HTML}{7F7F7F}
\definecolor{cblue}{HTML}{00A2E8}
\definecolor{cgreen}{HTML}{22B14C}
\definecolor{cyellow}{HTML}{FFF200}
\definecolor{corange}{HTML}{EA7904}
\definecolor{cpurple}{HTML}{9100FC}

\usepackage[most]{tcolorbox}
\tcbset{%
  on line,
  boxsep=4pt, left=0pt,right=0pt,top=0pt,bottom=0pt,
  arc=7.5pt,
  colframe=white,
  %colback=LightLavender,
  highlight math style={enhanced}
}
\renewcommand{\familydefault}{\sfdefault}

% The following chemfig code was generated via mol2chemfig:
% Glucose 6-phosphate
% mol2chemfig -w -i direct "C(C1C(C(C(C(O1)O)O)O)O)OP(=O)(O)O"
% Fructose 6-phosphate
% mol2chemfig -w -i direct "C(C(C(C(C(=O)CO)O)O)O)OP(=O)(O)O"
% Fructose 1,6-bisphosphate
% mol2chemfig -w -i direct "C(C1C(C(C(O1)(COP(=O)(O)O)O)O)O)OP(=O)(O)O"
% G3P
% mol2chemfig -w -i direct "C(C(C=O)O)OP(=O)(O)O"
% DHAP
% mol2chemfig -w -i direct "C(C(=O)COP(=O)(O)O)O"
% 1,3-biphosphoglycerate
% mol2chemfig -w -i direct "C(C(C(=O)OP(=O)(O)O)O)OP(=O)(O)O"

\begin{document}

%\setchemfig{atom sep=10pt}
\setchemfig{%
  scheme debug=false,
  atom style={scale=0.5},
  atom sep=2em,
  bond offset=1pt
}

\begin{tikzpicture}
  \node (G6P)
    []
    {%
      %\chemname{%
        \chemfig{%
                      OH% 8
             -[:90,,1]% 6
               %-[:150]% 5
               -[:150]{\colorbox{corange}{\color{black}{C}}}
                         (%
               -[:210,,,2]HO% 9
                         )
                %-[:90]% 4
                -[:90]{\colorbox{cblue}{\color{black}{C}}}
                         (%
               -[:150,,,2]HO% 10
                         )
                %-[:30]% 3
                -[:30]{\tcbox[colback=cblue]{C}}
                         (%
                -[:90,,,1]OH% 11
                         )
               %-[:330]% 2
               -[:330]{\tcbox[colback=corange]{C}}
                         (%
                   -[:270]O% 7
                   %-[:210]% -> 6
                   -[:210]{\colorbox{cgreen}{\color{black}{C}}}
                         )
                %-[:30]% 1
                -[:30]{\tcbox[colback=cgreen]{C}}
               -[:330]O% 12
                -[:30]P% 13
                         (%
                    =[:90]O% 14
                         )
                         (%
               -[:330,,,1]OH% 15
                         )
            -[:30,,,1]OH% 16
        }
      %}{G6P}
    };
  \node[below=0.1cm of G6P] () {G6P};
  \node (arr1)
    [right=0.1cm of G6P]
    {\schemestart\arrow{<=>}\schemestop};
  \node (F6P)
    [right=0.1cm of arr1]
    {%
      \chemfig{%
                     O% 6
               %=[:90]% 5
               =[:90]{\colorbox{corange}{\color{black}{C}}}
                        (%
                  %-[:150]% 7
                  -[:150]{\colorbox{cgreen}{\color{black}{C}}}
              -[:210,,,2]HO% 8
                        )
               %-[:30]% 4
               -[:30]{\colorbox{cblue}{\color{black}{C}}}
                        (%
               -[:90,,,1]OH% 9
                        )
              %-[:330]% 3
              -[:330]{\tcbox[colback=cblue]{C}}
                        (%
              -[:270,,,1]OH% 10
                        )
               %-[:30]% 2
               -[:30]{\tcbox[colback=corange]{C}}
                        (%
               -[:90,,,1]OH% 11
                        )
              %-[:330]% 1
              -[:330]{\tcbox[colback=cgreen]{C}}
               -[:30]O% 12
              -[:330]P% 13
                        (%
                  =[:270]O% 14
                        )
                        (%
               -[:30,,,1]OH% 15
                        )
          -[:330,,,1]OH% 16
      }
    };
  \node[below=0.1cm of F6P] () {F6P};
  \node (arr2)
    [right=0.1cm of F6P]
    {\schemestart\arrow{-U>[\tiny ATP][\tiny ADP][][0.25]}\schemestop};
  \node (F16P)
    [right=0.1cm of arr2]
    {%
      \chemfig{%
                   O% 10
            =[:228]P% 9
                      (%
            -[:168,,,2]HO% 11
                      )
                      (%
            -[:108,,,2]HO% 12
                      )
            -[:288]O% 8
            %-[:348]% 7
            -[:348]{\tcbox[colback=cgreen]{C}}
            %-[:288]% 5
            -[:288]{\tcbox[colback=corange]{C}}
                      (%
            -[:192,,,2]HO% 13
                      )
            %-[:276]% 4
            -[:276]{\tcbox[colback=cblue]{C}}
                      (%
            -[:222,,,2]HO% 14
                      )
            %-[:348]% 3
            -[:348]{\colorbox{cblue}{\color{black}{C}}}
                      (%
            -[:294,,,1]OH% 15
                      )
             %-[:60]% 2
             -[:60]{\colorbox{corange}{\color{black}{C}}}
                      (%
                -[:132]O% 6
                -[:204]{\tcbox[colback=corange]{C}}% -> 5
                      )
              %-[:6]% 1
              -[:6]{\colorbox{cgreen}{\color{black}{C}}}
             -[:66]O% 16
              -[:6]P% 17
                      (%
                =[:306]O% 18
                      )
                      (%
             -[:66,,,1]OH% 19
                      )
          -[:6,,,1]OH% 20
      }
    };
  \node[below=0.1cm of F16P] () {FDP};

  \node (G3P)
    [right=1.0cm of F16P, yshift=1.25cm]
    {%
      \chemfig{%
                    O% 4
              %=[:30]% 3
              =[:30]{\colorbox{cblue}{\color{black}{C}}}
             %-[:330]% 2
             -[:330]{\colorbox{corange}{\color{black}{C}}}
                       (%
             -[:270,,,1]OH% 5
                       )
              %-[:30]% 1
              -[:30]{\colorbox{cgreen}{\color{black}{C}}}
             -[:330]O% 6
              -[:30]P% 7
                       (%
                 =[:330]O% 8
                       )
                       (%
              -[:90,,,1]OH% 9
                       )
          -[:30,,,1]OH% 10
      }
    };
  \node[above=0.1cm of G3P] () {G3P};
  \node (DHAP)
    [right=1.0cm of F16P, yshift=-1.25cm]
    {%
      \chemfig{%
                   O% 3
             %=[:90]% 2
             %=[:90]{\colorbox{corange}{\color{black}{C}}}
             =[:90]{\tcbox[colback=corange]{C}}
                      (%
                 %-[:30]% 1
                 %-[:30]{\colorbox{cblue}{\color{black}{C}}}
                 -[:30]{\tcbox[colback=cblue]{C}}
            -[:330,,,1]OH% 10
                      )
            %-[:150]% 4
            %-[:150]{\colorbox{cgreen}{\color{black}{C}}}
            -[:150]{\tcbox[colback=cgreen]{C}}
            -[:210]O% 5
            -[:150]P% 6
                      (%
                =[:210]O% 7
                      )
                      (%
             -[:90,,,1]OH% 8
                      )
        -[:150,,,2]HO% 9
    }
  };
  \node[below=0.1cm of DHAP] () {DHAP};

  \node (arr3) at ($(G3P)!0.5!(DHAP)$)
    [rotate=90]
    {\schemestart\arrow{<=>}\schemestop};
  \draw[-stealth] (F16P) -- ( $ (F16P.0)!0.5!(G3P.west|-F16P.0) $ ) |- (G3P.west) node[auto,pos=0.7] {};
  \draw[-stealth] (F16P) -- ( $ (F16P.0)!0.5!(DHAP.west|-F16P.0) $ ) |- (DHAP.west) node[auto,pos=0.7] {};


\end{tikzpicture}


\end{document}

