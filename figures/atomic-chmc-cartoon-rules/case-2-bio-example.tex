\documentclass{standalone}
\usepackage[margin=1.00in]{geometry}
\usepackage{xcolor}
\usepackage{caption}
\usepackage{subcaption}
%\captionsetup[subfigure]{justification=justified,singlelinecheck=false}
\usepackage{chemfig}
\usepackage{mol2chemfig}

% Define colours
\definecolor{cred}{HTML}{ED1C24}
\definecolor{cgrey}{HTML}{7F7F7F}
\definecolor{cblue}{HTML}{00A2E8}
\definecolor{cgreen}{HTML}{22B14C}
\definecolor{cyellow}{HTML}{FFF200}
\definecolor{corange}{HTML}{EA7904}
\definecolor{cpurple}{HTML}{9100FC}
\renewcommand{\familydefault}{\sfdefault}

% The following chemfig code was generated via mol2chemfig:
% Glutathione peroxidase
% 2 GSH[m] + H2O2[m] => GSSG[m] + 2 H2O[m]
% mol2chemfig -w -i direct "C(CC(=O)NC(CS)C(=O)NCC(=O)O)C(C(=O)O)N"
% mol2chemfig -w -i direct "OO"
% mol2chemfig -w -i direct "C(CC(=O)NC(CSSCC(C(=O)NCC(=O)O)NC(=O)CCC(C(=O)O)N)C(=O)NCC(=O)O)C(C(=O)O)N"
% mol2chemfig -w -i direct "O"

\begin{document}

%\setchemfig{atom sep=10pt}
\setchemfig{%
  scheme debug=false,
  atom style={scale=0.5},
  atom sep=2em,
  bond offset=1pt
}

\begin{tikzpicture}
    \node (glut1)
        []
        {
    \chemfig{%
                   O% 4
             =[:90]% 3
                      (%
                 -[:30]% 2
                 -[:330]{\colorbox{cblue}{\color{black}{C}}}% 1
                 -[:30]% 16
                          (%
                 -[:90,,,1]NH_2% 20
                          )
                -[:330]% 17
                          (%
                 -[:30,,,1]OH% 19
                          )
                =[:270]O% 18
                      )
            -[:150]\mcfabove{N}{H}% 5
            -[:210]% 6
                      (%
                -[:270]% 7
            -[:330,,,1]SH% 8
                      )
            -[:150]% 9
                      (%
                 =[:90]O% 10
                      )
            -[:210]\mcfbelow{N}{H}% 11
            -[:150]% 12
            -[:210]% 13
                      (%
                =[:270]O% 14
                      )
        -[:150,,,2]HO% 15
    }
        };

        \node (p1)
            [below=1.0cm of glut1]
            {
        \schemestart\ \+ \schemestop
            };
    \node (glut2)
        [below=0.1cm of p1]
        {
    \chemfig{%
                   O% 4
             =[:90]% 3
                      (%
                 -[:30]% 2
                 -[:330]{\colorbox{cblue}{\color{black}{C}}}% 1
                 -[:30]% 16
                          (%
                 -[:90,,,1]NH_2% 20
                          )
                -[:330]% 17
                          (%
                 -[:30,,,1]OH% 19
                          )
                =[:270]O% 18
                      )
            -[:150]\mcfabove{N}{H}% 5
            -[:210]% 6
                      (%
                -[:270]% 7
            -[:330,,,1]SH% 8
                      )
            -[:150]% 9
                      (%
                 =[:90]O% 10
                      )
            -[:210]\mcfbelow{N}{H}% 11
            -[:150]% 12
            -[:210]% 13
                      (%
                =[:270]O% 14
                      )
        -[:150,,,2]HO% 15
    }
        };
    \node (h2o2)
        [right=0.1cm of p1]
        {
          \chemfig{%
                      HO% 1
              -[,,2,1]OH% 2
            }
        };

    \node (arr1)
    [right=0.1cm of h2o2]
    {
        \schemestart\ \arrow{<=>[Glutathione peroxidase][(Mitochondria)]}[0, 2.5]\schemestop\
    };

    \node (glutdi)
        [right=0.1cm of arr1]
        {
    \chemfig{%
                   O% 4
            =[:270]% 3
                      (%
                -[:330]% 2
                 -[:30]{\colorbox{corange}{\color{black}{C}}}% 1
                -[:330]% 36
                          (%
                -[:270,,,1]NH_2% 40
                          )
                 -[:30]% 37
                          (%
                -[:330,,,1]OH% 39
                          )
                 =[:90]O% 38
                      )
            -[:210]\mcfbelow{N}{H}% 5
            -[:150]% 6
                      (%
                -[:210]% 29
                          (%
                    -[:150]\mcfabove{N}{H}% 31
                    -[:210]% 32
                    -[:150]% 33
                              (%
                    -[:210,,,2]HO% 35
                              )
                     =[:90]O% 34
                          )
                =[:270]O% 30
                      )
             -[:90]% 7
             -[:30]S% 8
             -[:90]S% 9
             -[:30]% 10
             -[:90]% 11
                      (%
                 -[:30]% 12
                          (%
                 -[:90,,,2]HN% 14
                  -[:30,,2]% 15
                     -[:90]% 16
                              (%
                     -[:30,,,1]OH% 18
                              )
                    =[:150]O% 17
                          )
                =[:330]O% 13
                      )
        -[:150,,,2]HN% 19
          -[:90,,2]% 20
                      (%
                 =[:30]O% 21
                      )
            -[:150]{\colorbox{cblue}{\color{black}{C}}}% 22
             -[:90]% 23
            -[:150]% 24
                      (%
            -[:210,,,2]H_2N% 28
                      )
             -[:90]% 25
                      (%
                 =[:30]O% 26
                      )
        -[:150,,,2]HO% 27
    }
        };

        \node (p2)
            [right=0.1cm of glutdi]
            {
        \schemestart\ \+ \schemestop
            };
        \node (h2o)
            [right=0.1cm of p2]
            {
              \chemfig{%
                  H_2O% 1
              }
            };
\end{tikzpicture}

\end{document}

