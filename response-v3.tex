\documentclass[11pt]{article}
\usepackage[margin=1.0in]{geometry}
\usepackage{xcolor}

\begin{document}

\noindent Dear Editors and Reviewers,\\

We thank both the editors and reviewers for their feedback on our
manuscript and have revised it accordingly. We detail the changes below,
but the main changes are as follows. First, we have revised our
introduction to describe related approaches for constructing and analysing
metabolic networks at the atomic level. These additions clarify how our
analysis of atomic transition graphs via AEFMs is distinct from other types of
pathway analysis methods. We further discuss the relationship between AEFMs and
EFMs and why these atomic pathways can be viewed as the EFMs of a certain
atomic flux graph. Second, we have described our computational pipeline in
greater detail in the methods section. We explain the construction and analysis
of the ACHMC to obtain AEFM probabilities that are scaled to AEFM weights.
Finally, we have noted two atom mapping errors in our HepG2 weight
analysis from RXNMapper. We describe these two erroneous reaction mappings
in-text and have updated all figures, tables, and statistics. In
particular, our final HepG2 figure shows dominant AEFMs that are not only
more biologically accurate, but predict greater metabolic activity through
the non-canonical TCA cycle pathway.

~

\noindent Sincerely,

\noindent Justin G. Chitpin, Theodore J. Perkins

%\noindent\textcolor{blue}{\hline} 

%\hline

~\\

\noindent\textcolor{blue}{Reviewer: 1}\\

No comments.\\

\noindent\textcolor{blue}{Reviewer: 2}\\

\noindent\textcolor{blue}{Major comments:}\\

\noindent\textcolor{blue}{1. The reviewer was very puzzled by the proposed
approach and lack of any mention of previous studies that have precisely
proposed the same idea. For instance,\\
https://pmc.ncbi.nlm.nih.gov/articles/PMC2881407/ and\\
https://bmcbioinformatics.biomedcentral.com/articles/10.1186/1471-2105-8-315
to name just a few.}\\

These papers, as well as several others that analyse various notions of
atomic transition graphs derived from metabolic flux networks, are
discussed in our introduction. We have included a new figure (Fig 1) that
visualises these different types of networks and explains how these
constructs are analysed to uncover metabolic pathways or estimate
intracellular fluxes. Although some elements of our approach have been
previously proposed, the pipeline we offer has unique distinguishing
strengths not present in any single prior work: (1) We automatically
(albeit with manual checking) construct atomic flux networks based on
SMILES descriptions of metabolites and reaction stoichiometries, using
a machine-learning based approach; (2) we have generalised previous atomic
flux methods to work with any element (not just carbon, but also nitrogen
or any other element, except for hydrogen, which is not well-handled by
the RXNMapper software, and is any case often omitted from metabolite
descriptions); (3) use serial or parallel computing strategies to
efficiently compute AEFMs for different source atoms; (4) we can
\textit{uniquely} decompose atomic fluxes onto the AEFMs using our
previously-published Markob chain-based theory. With these combined
strengths, our paper demonstrates for the first time the feasibility of
flux decomposition on genome-scale metabolic networks.

\noindent\textcolor{blue}{2. The concept of AEFM has nothing to do with EFMs.
The so-called AEFMs are paths in a graph, given by the atom mapping.
A metabolic network, in contrast, is a hypergraph – precluding the search for
(shortest) paths. Therefore, the naming of the paths itself gives a wrong
impression that the concept has anything to do with EFMs.}\\

Following point 1, we better describe the concept of the AEFM and its
precise relationship to EFMs in our introduction. Briefly, the formal
relationship is that AEFMs \textit{are} EFMs of the atomic flux graph of
a metabolic network. Similar terminology exists in literature from Pey
\textit{et al.} (2011) who propose ``elementary carbon modes" to describe
what we could call carbon AEFMs. Because Pey's terminology does not
readily generalise to other element types, we believe AEFM is a suitable
and appropriate term that accurately describes our proposed pathway
concept. We recognise, and emphasise in the introduction, that AEFMs are
\textit{not} EFMs of the metabolic flux network. We demonstrate this with
additional panels in Fig 1.\\

\noindent\textcolor{blue}{3. To make point 2 more precise, presence of any
splitting or condensation reaction would result in calling a sequence of
reactions (i.e. a path) an AEFM although these reactions are only subsets
of an EFM and cannot themselves support steady-state.}\\

We agree that an AEFM in a multispecies network does not correspond to
a minimal pathway maintaining steady state reaction flux in a metabolic
flux network. Our new figure comparing EFMs/AEFMs highlights this
distinction between the two pathway types and how EFMs balance molecular
stoichiometries across reactions while AEFMs balance individual atomic
movements.\\

\noindent\textcolor{blue}{4. Therefore, the study presents a comparison of
``apples and oranges'', since the two concepts cannot be reconciled.}\\

EFMs and AEFMs solve different, but related problems: Giving
insight into flux pathways is complicated metabolic networks. Similarly,
one might compare a bicycle and a car as alternate modes of
transportation---while recognising that they do not solve entirely the
same set of problems. For that matter, one might compare the relative
merits of apples and oranges as part of one's diet, without claiming that
apples and oranges are interchangeable. We do not claim that EFMs and
AEFMs solve exactly the same problem, and indeed this is a strength of our
paper. By exploring AEFMs as an alternative to EFMs, we have been able to
demonstrate the previously-unrecognised fact that AEFM enumeration in
atomic flux networks is computationally \textit{more feasible} than EFM
enumeration in the corresponding molecular flux networks -- at least for
the five different networks we studied.\\

\noindent\textcolor{blue}{5. As a result of this elaboration, the
CHMC-based approach is not applicable in this setting, contrary to what is
claimed on page 6, last paragraph.}\\

Recognising that AEFMs are EFMs of atomic flux graphs, and that atomic
flux graphs are unimolecular transformation networks as studied in our
previous CHMC work, the ACHMC approach we describe here is applicable and
mathematically sound.\\

\noindent\textcolor{blue}{6. The authors also draw analogies to
isotopomers from “metabolite revisitation”, which also is misleading and
erroneous; isotopomers are properties of metabolites irrespective of the
network in which they occur (i.e. they are independent of the network
structure).}\\

We thank the Reviewer for pointing this out and the misleading statement has
been removed.\\

\noindent\textcolor{blue}{Other revisions:}\\

\noindent\textcolor{blue}{Figure formatting:}\\

We have improved figure font readability and uniformity across our figures. In
particular, Figure S1 has been split into two figures.\\

\noindent\textcolor{blue}{Description of the methods:}\\

We structure the methods section to be more consistent with our AEFM pipeline
figure. We describe the atomic flux graph in more detail and how our ACHMC is
constructed from atomic flux graphs specific to each source
metabolite-atom position. We further provide pseudocode for constructing
the ACHMC. Finally, we describe how the ACHMC is analysed to identify
simple cycles corresponding to AEFMs and how AEFM probabilities and
weights are computed from these simple cycles.

\end{document}

